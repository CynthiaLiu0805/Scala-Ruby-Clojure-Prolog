% Created 2020-11-26 Thu 20:35
% Intended LaTeX compiler: pdflatex
\documentclass[11pt]{article}
\usepackage[utf8]{inputenc}
\usepackage[T1]{fontenc}
\usepackage{graphicx}
\usepackage{grffile}
\usepackage{longtable}
\usepackage{wrapfig}
\usepackage{rotating}
\usepackage[normalem]{ulem}
\usepackage{amsmath}
\usepackage{textcomp}
\usepackage{amssymb}
\usepackage{capt-of}
\usepackage{hyperref}
\usepackage{unicode-math}
\usepackage{unicode}
\author{Cynthia Liu}
\date{\today}
\title{Computer Science 3MI3 – 2020 Assignment 2: Typing a λ-calculus}
\hypersetup{
 pdfauthor={Cynthia Liu},
 pdftitle={Computer Science 3MI3 – 2020 Assignment 2: Typing a λ-calculus},
 pdfkeywords={},
 pdfsubject={Assignment 2 in Scala and Ruby implementation},
 pdfcreator={Emacs 27.1 (Org mode 9.3)}, 
 pdflang={English}}
\begin{document}

\maketitle
\tableofcontents


\section{Introduction}
\label{sec:org4a04684}
This is the documentation for Assignment 2 for COMPSCI3MI3 2020fall. 

It is about constructing representation of a simply-typed λ-calculus, 
and a typechecker for that λ-calculus, as well as a type-erasure and a simple translator to simplify terms to untyped λ-calculus terms.

This Assignment is written in both Scala and Ruby. 

\section{Part one: The representation}
\label{sec:org5c2a176}
\subsection{Representation in Scala}
\label{sec:org768944a}
The terms of the λ-calculus ST are expressed in the new type \texttt{STTerm}.
\begin{itemize}
\item The constructor \texttt{STVar} take a parameter of type \texttt{Int}.
\item The constructor \texttt{STZero}, \texttt{STTrue}, \texttt{STFalse} take no arguments, so they are implemented as case object
\item The constructor \texttt{STSuc}, \texttt{STIsZero} take one parameters of type \texttt{STTerm}.
\item The constructor \texttt{STApp} takes two parameters of type \texttt{STTerm}.
\item The constructor \texttt{STAbs} takes two parameters, one is of type \texttt{STTerm}, the other is of type \texttt{STType}.
\item The constructor \texttt{STTest} takes three parameters of type \texttt{STTerm}.
\end{itemize}


\begin{verbatim}
sealed trait STTerm
case class STVar(index: Int) extends STTerm
case class STApp(t1: STTerm,t2: STTerm) extends STTerm
case class STAbs(t: STType,term: STTerm) extends STTerm

case object STZero extends STTerm
case class STSuc(t: STTerm) extends STTerm
case class STIsZero(t: STTerm) extends STTerm

case object STTrue extends STTerm
case object STFalse extends STTerm
case class STTest(t1: STTerm,t2: STTerm,t3: STTerm) extends STTerm

\end{verbatim}


\subsection{Representation in Ruby}
\label{sec:org5bd2398}
In Ruby, all constructors are subtypes of type \texttt{STTerm}, that is, they all inherites from \texttt{STTerm}
\begin{verbatim}
class STTerm end
\end{verbatim}

The operator \texttt{<} means inheritance, for each class, we have an initialize method to initialize the object, and we have a == method to do comparsion.
\begin{itemize}
\item For \texttt{STVar}, it is initialized with an integer.
\end{itemize}
\begin{verbatim}
class STVar < STTerm
  attr_reader :index
  # We require our variables are only indexed by integers.
  def initialize(index)
    unless index.is_a?(Integer) 
      throw "Constructing a STVar out of non-integer terms"
    end
    @index = index
  end
  def ==(type); type.is_a?(STVat) && type.index==@index end
end
\end{verbatim}

\begin{itemize}
\item For \texttt{STZero}, \texttt{STTrue}, \texttt{STFalse}, they are initialized directly.
\end{itemize}
\begin{verbatim}
class STZero < STTerm
  def ==(type); type.is_a?(STZero) end
  def to_s; "zero" end
  def typeOf(arr); STNat.new end
end
\end{verbatim}

\begin{itemize}
\item For \texttt{STSuc}, \texttt{STIsZero}, they are initialized with a \texttt{STTerm}.
\end{itemize}
\begin{verbatim}
class STSuc < STTerm
    attr_reader :t
    def initialize(t)
      unless t.is_a?(STTerm)
	throw "Constructing a lambda term out of non-lambda terms"
      end
      @t = t
    end

    def ==(type); type.is_a?(STSuc) && type.t==@t end
end
\end{verbatim}

\begin{itemize}
\item For \texttt{STApp}, it is initialized with two \texttt{STTerm}.
\end{itemize}
\begin{verbatim}
class STApp < STTerm
    attr_reader :t1
    attr_reader :t2
    def initialize(t1,t2)
      unless t1.is_a?(STTerm) && t2.is_a?(STTerm)
	throw "Constructing a lambda term out of non-lambda terms"
      end
      @t1 = t1; @t2 = t2
    end
    def ==(type); type.is_a?(STApp) && type.t1==@t1 && type.t2==@t2 end
  end  
\end{verbatim}

\begin{itemize}
\item For \texttt{STAbs}, it is initialized with a \texttt{STTerm} and a \texttt{STType}.
\end{itemize}
\begin{verbatim}
class STAbs < STTerm
    attr_reader :t1
    attr_reader :t2
    def initialize(t1,t2)
      unless t1.is_a?(STType) && t2.is_a?(STTerm)
	throw "Constructing a lambda term out of non-lambda terms"
      end
      @t1 = t1; @t2 = t2
    end

    def ==(type); type.is_a?(STAbs) && type.t1==@t1 && type.t2==@t2 end
end
\end{verbatim}

\begin{itemize}
\item For \texttt{STTest}, it is initialized with three \texttt{STTerm}.
\end{itemize}
\begin{verbatim}
class STTest < STTerm
   attr_reader :t1
   attr_reader :t2
   attr_reader :t3

   # We require our variables are only indexed by integers.
   def initialize(t1,t2,t3)
     unless t1.is_a?(STTerm) && t2.is_a?(STTerm) && t2.is_a?(STTerm)
       throw "Constructing a lambda term out of non-lambda terms"
     end
     @t1 = t1; @t2 = t2; @t3=t3
   end

   def ==(type); type.is_a?(STTest) && type.t1==@t1 && type.t2==@t2 && type.t3==@t3 end
 end
\end{verbatim}


\section{Part two: Typechecking}
\label{sec:org75fe152}
This method takes a \texttt{STTerm}, and returns true if the represented term obeys the type rules of ST; 
otherwise, it returns false.

\subsection{Scala implementation}
\label{sec:org94eedd4}
\subsubsection{typeOf}
\label{sec:org80efad3}
This method determine a type for the input \texttt{STTerm}. 
It has two arguments, one is the \texttt{STTerm}, the other one is an typing context, here an empty list is used.
It return a \texttt{Option[STType]}.

\begin{itemize}
\item Acccording to the typing rule, for \texttt{STVar}, the type is given by the environment
\end{itemize}
\begin{verbatim}
case STVar(index) => None                                                                                                                                                                                                                                             
\end{verbatim}

\begin{itemize}
\item For \texttt{STTrue}, \texttt{STFalse}, the type is \texttt{STBool}
\end{itemize}
\begin{verbatim}
case STTrue | STFalse => Some(STBool)
\end{verbatim}

\begin{itemize}
\item For \texttt{STIsZero}, if the type of its parameter t is \texttt{STNat}, then the type of \texttt{STBool}
\end{itemize}
\begin{verbatim}
case STIsZero(t) => if (typeOf(t,List[STType]())==Some(STNat)) Some(STBool) else None
\end{verbatim}

\begin{itemize}
\item For \texttt{STZero}, the type is \texttt{STNat}
\item For \texttt{STSuc}, if the type of its parameter t is \texttt{STNat}, then the type is \texttt{STNat}
\end{itemize}
\begin{verbatim}
case STZero => Some(STNat)
case STSuc(t) => if (typeOf(t,List[STType]())==Some(STNat)) Some(STNat) else None
\end{verbatim}

\begin{itemize}
\item For \texttt{STAbs}, we add the type to the list(environment) first, if the index(index parameter of the \texttt{STVar}) at the list is of type t, return A -> A.

If it is a free variable, return None. If it is not a \texttt{STVar}, use recursion to find the type of the term
\end{itemize}
\begin{verbatim}
case STAbs(t,term) => 
  term match {    
    case STVar(index) if ((l:+(t)).length < index) => print((t::l).length< index); None
    case STVar(index) if (((l:+(t)).lift(index).get)==t) => Some(STFun(t,STNat))
    case STApp(t1, t2) => ((l:+(t)).lift(0).get) match {case STFun(dom,codom) if (dom==t) => print(t); Some(STNat)
    case _ => Some(STFun(t,typeOf(term,l:+(t)).get))
  }
\end{verbatim}

\begin{itemize}
\item For \texttt{STApp}, we add the type of t1 to the list and get the first element of the list.

If it is of type A -> B, check if type1 is the same as type of t2, if so, the general type is type2, if not, return None.
\end{itemize}
\begin{verbatim}
case STApp(t1, t2) => (typeOf(t1,List[STType]())::l)(0) match {
    case Some(STFun(type1,type2)) if Some(type1)==typeOf(t2,List[STType]()) => print((typeOf(t1,List[STType]())::l)(0));print(type2);Some(type2)
    case _ => None 
  } 
\end{verbatim}


\subsubsection{typecheck}
\label{sec:org06f907c}
This method takes an \texttt{STTerm}, and returns \texttt{true} if the represented term obeys the type rules of ST; 
otherwise, it returns false.

The \texttt{typeOf} method is called in this method, if it return some type, the result is true, else it is false. 
Also, if exception occurs, it is false
\begin{verbatim}
def typecheck(input:STTerm):Boolean= try {
  if (typeOf(input, List[STType]())==None) {
    return false
  }
  else {
    return true
  }
}
  catch {
      case _: Throwable => false
  }
\end{verbatim}


\subsection{Ruby implementation}
\label{sec:org21db10b}
\subsubsection{typeOf}
\label{sec:org5bd96e5}
This method determine a type for the input \texttt{STTerm}. 
As it is implemented in each class, an empty environment is passed as argument. Here, an empty array is used.

\begin{itemize}
\item For \texttt{STVar}, the type is given by the environment
\end{itemize}
\begin{verbatim}
def typeOf(arr); nil end
\end{verbatim}

\begin{itemize}
\item For \texttt{STTrue}, \texttt{STFalse}, the type is \texttt{STBool}
\end{itemize}
\begin{verbatim}
def typeOf(arr); STBool.new end
\end{verbatim}

\begin{itemize}
\item For \texttt{STIsZero}, if the type of its parameter t is \texttt{STNat}, then the type of \texttt{STBool}
\end{itemize}
\begin{verbatim}
def typeOf(arr)
      if t.typeOf(Array.new)==STNat.new
	return STBool.new
      else 
	return nil
      end  
    end
\end{verbatim}

\begin{itemize}
\item For \texttt{STZero}, the type is \texttt{STNat}
\end{itemize}
\begin{verbatim}
def typeOf(arr); STNat.new end
\end{verbatim}

\begin{itemize}
\item For \texttt{STAbs}, if t2 is of type \texttt{STVar}, we add the type to the list(environment) first, and the index(index parameter of the \texttt{STVar}) at the list is of type t1, return A -> A.

If it is a free variable, return None. If it is not a \texttt{STVar}, use recursion to find the type of the term
\end{itemize}
\begin{verbatim}
def typeOf(arr)
  case t2
  when STVar
    if (arr<<t1)[t2.index]==t1
      return STFun.new(t1,STNat.new)
    else 
      return nil
    end
  when STApp
    if (arr<<t1)[0].dom == t1
      return STNat.new
    end
  else 
    return STFun.new(t1,t2.typeOf(arr<<t1))
  end
end
\end{verbatim}

\begin{itemize}
\item For \texttt{STApp}, we check the type of t1, if it is \texttt{STFun}, we then check if its dom is the same as the type of t2.

If so, return its condom, else, return nil
\end{itemize}
\begin{verbatim}
def typeOf(arr)
      case t1.typeOf(Array.new)
      when STFun
	if t1.typeOf(Array.new).dom == t2.typeOf(Array.new)
	  return t1.typeOf(Array.new).codom
	else
	  return nil
	end
      else 
	return nil
      end
    end
\end{verbatim}


\subsubsection{typecheck}
\label{sec:orgb829cfc}
This method takes an \texttt{STTerm}, and returns \texttt{true} if the represented term obeys the type rules of ST; 
otherwise, it returns false.

The \texttt{typeOf} method is called in this method, if it return some type, the result is true, else it is false. 
Also, as this method applies to all class, it is defined in the super class STTerm
\begin{verbatim}
class STTerm 
  def typecheck
    if typeOf(Array.new)==nil
      return false
    else
      return true
    end
  end
end
\end{verbatim}


\section{Part three: Translation to the untyped λ-calculus}
\label{sec:org51cb0eb}
This method translates a STTerm into elements of ULTerm.

\subsection{Scala implementation}
\label{sec:org621452c}
\begin{itemize}
\item For \texttt{STVar}, \texttt{STTrue}, \texttt{STFalse}, \texttt{STZero} they are translated directly
\end{itemize}
\begin{verbatim}
case STVar(index) => ULVar(index)
case STTrue => ULAbs(ULAbs(ULVar(1)))
case STFalse => ULAbs(ULAbs(ULVar(0)))
case STZero => ULAbs(ULAbs(ULVar(0)))
\end{verbatim}

\begin{itemize}
\item For \texttt{STSuc(t)}, the \texttt{Suc} part is default, and \texttt{ULApp} is used to combine it with t in untyped lambda form.
\end{itemize}
\begin{verbatim}
case STSuc(t) => ULApp(ULAbs(ULAbs(ULAbs(ULApp(ULVar(1),ULApp(ULApp(ULVar(2),ULVar(1)),ULVar(0)))))),eraseTypes(t))
\end{verbatim}

\begin{itemize}
\item For \texttt{STApp} and \texttt{STAbs}, they are translated using recursion
\end{itemize}
\begin{verbatim}
case STApp(t1, t2) => ULApp(eraseTypes(t1),eraseTypes(t2))
case STAbs(t, term) => ULAbs(eraseTypes(term))
\end{verbatim}


\subsection{Ruby implementation}
\label{sec:org5358304}
\begin{itemize}
\item For \texttt{STVar}, \texttt{STTrue}, \texttt{STFalse}, \texttt{STZero} they are translated directly
\end{itemize}
\begin{verbatim}
def eraseTypes; ULVar.new(index) end
def eraseTypes; ULAbs.new(ULAbs.new(ULVar.new(1))) end
def eraseTypes; ULAbs.new(ULAbs.new(ULVar.new(0))) end
def eraseTypes; ULAbs.new(ULAbs.new(ULVar.new(0))) end
\end{verbatim}

\begin{itemize}
\item For \texttt{STSuc(t)}, the \texttt{Suc} part is default, and \texttt{ULApp} is used to combine it with t in untyped lambda form.
\end{itemize}
\begin{verbatim}
def eraseTypes
    return ULApp.new(ULAbs.new(ULAbs.new(ULAbs.new(ULApp.new(
      ULVar.new(1),ULApp.new(ULApp.new(ULVar.new(2),ULVar.new(1)),ULVar.new(0)))))),t.eraseTypes) 
end
\end{verbatim}

\begin{itemize}
\item For \texttt{STApp} and \texttt{STAbs}, they are translated using recursion
\end{itemize}
\begin{verbatim}
def eraseTypes; ULApp.new(t1.eraseTypes,t2.eraseTypes) end
def eraseTypes; ULAbs.new(t2.eraseTypes) end
\end{verbatim}
\end{document}
